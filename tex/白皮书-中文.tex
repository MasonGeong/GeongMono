\documentclass[12pt, a4paper]{ctexart}

% --- 宏包引入 ---
\usepackage{fontspec} % 字体管理
\usepackage{graphicx} % 插入图片
\usepackage{geometry} % 页面设置
\usepackage{amsmath, amssymb} % 数学符号
\usepackage{booktabs} % 表格美化
\usepackage{tikz} % 绘图
\usepackage{hyperref} % 超链接
\usepackage{float}
\usepackage{cite} % 优化引用显示
\usepackage{pifont}

% --- 页面设置 ---
\geometry{left=2.5cm, right=2.5cm, top=3cm, bottom=3cm}

% --- 链接颜色设置 ---
\hypersetup{
    colorlinks=true,
    linkcolor=black,
    filecolor=magenta,      
    urlcolor=blue,
    citecolor=blue, % 引用数字的颜色
    pdftitle={Geong Numerals Whitepaper},
}

% --- 字体设置 (按需开启) ---
\newfontfamily\GeongFont{GeongNumMono-Regular.ttf}
% \setmonofont{JetBrains Mono} 
%\newfontfamily\GeongFont{Geong Mono}
% 伪命令,实际使用请取消上方注释并修改下方定义
\newcommand{\geong}[1]{{\GeongFont #1}}
% \newcommand{\geong}[1]{{\GeongFont #1}}

% --- 定义无编号脚注命令 ---
\newcommand\blfootnote[1]{%
  \begingroup
  \renewcommand\thefootnote{}\footnote{#1}%
  \addtocounter{footnote}{-1}%
  \endgroup
}

% --- 文档信息 ---
\title{\textbf{匠氏数字白皮书}\\ \large 重构数字符号:一种基于几何拓扑的直观表达系统}
\author{Mason Geong}
\date{\today}

\begin{document}

\maketitle

\begin{abstract}
本文提出了一种全新的数字表达系统——匠氏数字(Geong Numerals)。该系统旨在解决传统阿拉伯数字在视觉量感上的抽象性问题,通过结合几何图形的拓扑结构与二五混合进制逻辑(Bi-quinary logic),创造出一套兼具美学价值与算术直观性的符号。本文将详细阐述其设计哲学、排版美学规范、开源实现方案以及在不同进制下的拓展应用。
\end{abstract}
\blfootnote{本文档及匠氏符号逻辑已作为开源知识资产(Open Knowledge Assets)发布。欢迎开发者基于此协议进行二次创作或将其铭刻(Inscribe)于链上。支持与交流地址:\texttt{0x8D274F7ECcb7E3f0E191CFeB2D9954f963436f04}(ETH/Base (L2))。}
\newpage
\tableofcontents
\newpage

\section{引言}

\subsection{数字符号的直观性困境}
在人类数字文明中,符号的演变往往在“直观性”与“书写效率”之间权衡。现代通用的阿拉伯数字虽然书写高效,但其字形与数值之间失去了几何联系——符号“4”并不包含“4个单位”的视觉特征。

相比之下,玛雅数字(Mayan numerals)\cite{mayan} 与罗马数字保留了部分累加逻辑,而中国算盘(Suanpan)\cite{suanpan} 则通过位置与数量的结合达到了极致的高效。本文受到电影《异形》(Alien)\cite{alien} 中外星符号工业美学的启发,结合算盘的算理,提出了匠氏(Geong)数字,如表\ref{tab:num}所示。

\subsection{核心逻辑:二五混合制}
匠氏数字并非简单的五进制系统,而是基于 \textbf{二五混合码(Bi-quinary coded decimal)}\cite{bi-quinary} 的逻辑变体。它保留了十进制的位值制框架,但在单个位数的表达上,采用“5 + n”的结构。这种逻辑使得符号在视觉上即可直接进行无需记忆的加减运算。

\section{符号构造与书写}

\subsection{基础构造}
匠氏数字由两个核心元素构成:
\begin{itemize}
    \item \textbf{核心(Quinary Unit)}:使用圆圈( $\bigcirc$ )表示数值 5。
    \item \textbf{外围(Unary Units)}:使用包围核心的方框( $\square$ )线条表示数值 1 至 4。
\end{itemize}
这种设计使得 $6 \sim 9$ 成为 $1 \sim 4$ 的“加权”版本,形成了视觉上的对仗。

\subsection{零的哲学表达:空集与禁止}
在匠氏数字系统中,0 代表绝对的“无”。我们参考数学中的空集符号($\emptyset$)及交通标识中的“禁止”含义。
\begin{quote}
    \textbf{设计规范}:0 的字形由一个圆圈及穿过其中的斜杠构成。斜杠应当在视觉上呈现“不出头”的克制感,或仅保留内部线条,既区别于数字5($\bigcirc$),也区别于数字4( $\square$ )。
\end{quote}

\subsection{拓扑约束与笔画}
官方书写建议遵循顺时针原则(上$\rightarrow$右$\rightarrow$下$\rightarrow$左)。但在速记或手写体中,只要满足以下\textbf{拓扑连续性},即视为合法:
\begin{itemize}
    \item \textbf{合法}:笔画必须首尾相连或形成包围趋势(如折线)。
    \item \textbf{非法}:出现分离的笔画。例如,用平行的上下两横表示2是禁止的,因为这破坏了“方框”的封闭预期。
\end{itemize}

\begin{table}[H]
    \centering
    \caption{匠氏(Geong)数字与阿拉伯数字对照表}
    \vspace{0.3cm}
    \label{tab:num}
    % 定义表格列格式,中间一列专门用 Geong 字体显示
    \begin{tabular}{c c l}
        \toprule
        \textbf{数值} & \textbf{匠氏字形} & \textbf{几何构成} \\
        \midrule
        0 & \geong{0} & 空集变体(不出头 $\emptyset$) \\
        1 & \geong{1} & 单边(上) \\
        2 & \geong{2} & 双边折线(上+右) \\
        3 & \geong{3} & 三边折线(上+右+下) \\
        4 & \geong{4} & 封闭方框 \\
        5 & \geong{5} & 核心圆 \\
        6 & \geong{6} & 圆 + 单边 \\
        7 & \geong{7} & 圆 + 双边 \\
        8 & \geong{8} & 圆 + 三边 \\
        9 & \geong{9} & 圆 + 方框 \\
        \bottomrule
    \end{tabular}
\end{table}

\section{字体设计美学指南}

为了适应不同的排版场景,匠氏数字在字体设计上提供了两种风格建议。

\subsection{衬线体风格 (Serif Style)}
适用于正式文档、文学排版或复古科幻场景。
\begin{itemize}
    \item \textbf{连接方式}:线条转角处应采用严谨的\textbf{直角连接(Miter Join)},强调工业结构感。
    \item \textbf{装饰细节}:笔画的起止点或转角处可选配\textbf{衬线(Serif)}或微小的\textbf{饰线(Spur)},类似汉字明体或拉丁字体的衬线脚,以增加阅读的引导性。
\end{itemize}

\subsection{无衬线体风格 (Sans-serif Style)}
适用于现代UI界面、屏幕显示或代码编写。
\begin{itemize}
    \item \textbf{连接方式}:推荐使用\textbf{圆角连接(Round Join)},使方框线条呈现流畅的胶囊感。
    \item \textbf{笔触}:采用等宽线条,末端可做平头或圆头处理,保持极简主义风格。
\end{itemize}

\section{开源实现:Geong Mono}

为了验证该系统的实用性,我们发布了基于开源字体 \texttt{JetBrains Mono} \cite{jetbrains} 修改制作的字体——\textbf{Geong Mono}。

\subsection{实现细节}
该字体利用 OpenType 特性或直接替换字形码位(Code Points),将 ASCII 区段的 $0 \sim 9$ 映射为 Geong氏数字符号。
\begin{itemize}
    \item \textbf{底本}:JetBrains Mono(一款专为开发者设计的等宽字体)。
    \item \textbf{修改}:保留了原字体的字母设计,仅重绘了数字区段,确保了在代码环境下的可读性与对齐。
\end{itemize}

\subsection{获取方式}
源代码与字体文件已在 GitHub 开源:
\begin{center}
    \url{https://github.com/MasonGeong/GeongMono}
\end{center}

\section{进制拓展:以十六进制为例}

匠氏数字不仅限于十进制,其几何逻辑可以自然拓展至任何进制,下文以十六进制(Hexadecimal)为例。

\subsection{0x 拓展定义}
在十六进制模式下(前缀 \texttt{\geong{0}x}):
\begin{itemize}
    \item \textbf{中心圆}:代表权重 \textbf{8}($16 \div 2$)。
    \item \textbf{外围结构}:由四边形拓展为\textbf{正七边形},每条边代表 1。
\end{itemize}

\subsection{几何定向法则}
对于正七边形这种奇数边多边形,为了统一视觉重心,规定:
\textbf{底边(Bottom Edge)必须保持水平向右(0度)}。
这意味着 $0 \sim 7$ 的计数将围绕中心顺时针累加,确保每个数值的几何指向性唯一。

\section{可视化示例}
图\ref{fig:add}展示了使用匠氏计数系统在十进制下进行竖式加减法的过程。加减的过程可以通过图形直观地观察而无需记忆加减法表。
\begin{figure}[H]
    \centering
    \includegraphics[width=0.8\textwidth]{a-crop.pdf}
    \caption{使用匠氏计数系统进行竖式加减法}
    \label{fig:add}
\end{figure}
\newpage
% --- 参考文献部分 ---
\begin{thebibliography}{99}
    
    \bibitem{suanpan}
    Wikipedia contributors. "Suanpan (Abacus)." \textit{Wikipedia, The Free Encyclopedia}. 
    \url{https://en.wikipedia.org/wiki/Suanpan}
    
    \bibitem{bi-quinary}
    Wikipedia contributors. "Bi-quinary coded decimal." \textit{Wikipedia, The Free Encyclopedia}. 
    \url{https://en.wikipedia.org/wiki/Bi-quinary_coded_decimal}
    
    \bibitem{mayan}
    Wikipedia contributors. "Maya numerals." \textit{Wikipedia, The Free Encyclopedia}. 
    \url{https://en.wikipedia.org/wiki/Maya_numerals}
    
    \bibitem{alien}
    Giger, H. R. \textit{Necronomicon}. Sphinx Verlag, 1977. (Visual inspiration for Alien).
    
    \bibitem{jetbrains}
    JetBrains. "JetBrains Mono: A typeface for developers."
    \url{https://www.jetbrains.com/lp/mono/}
    
\end{thebibliography}

\end{document}